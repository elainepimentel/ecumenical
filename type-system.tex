%!TEX root = Ecumenical-types.tex
% !TEX spellcheck = en-US

 Ecumenical calculi should naturally mix the $\lambda_\mu$ internalization of {\em stoups} 
 and the continuation-passing aspect of general rules. 
 In this sense we propose an ecumenical term calculus. 
 \red{In this first draft, we will restrict ourselves to the implicational fragment.}

\subsection{Syntax and operational semantics}
\begin{definition}[Syntax]
Suppose given a denumerably infinite sets 
of {\em variables} $\set{\var,\vartwo,\hdots}$ 
and {\em names} $\set{\name,\nametwo,\hdots}$.
The set of {\em terms} $\set{\tm,\tmtwo,\hdots}$
and {\em commands} $\set{\ctm,\ctmtwo,\hdots}$
of the $\lambdamulep$-calculus are defined mutually inductively as follows.

\begin{marianaenv}
Terms:
\[
\begin{array}{lcl}
  \tm,\tmtwo,\tmthree 
         & ::=  & \var                               \\
         & \mid & \absi{\var}{\tm}                   \\
         & \mid & \appi{\tm}{\tmtwo}{\var}{\tmthree} \\ 
         & \mid & \absc{\var}{\name}{\ctm}           \\
         & \mid & \appc{\tm}{\tmtwo}{\var}{\ctm}     \\
         & \mid & \str{\ctm}
\end{array}
\]
\end{marianaenv}

Commands:
\[
\begin{array}{lcl}
  \ctm & ::=  & \command{\name}{\tm}
\end{array}
\]

\cyan{EP. Louis observed that $\appc{\tm}{\tmtwo}{\var}{\ctm} $ should be also a command. I agree, since the target term is a command.}


\begin{marianaenv}
This grammar has two constructors: $\absi{\var}{\tm}$ and $\absc{\var}{\name}{\ctm}$,
two generalized applications: $\appi{\tm}{\tmtwo}{\var}{\tmthree}$ and $\appc{\tm}{\tmtwo}{\var}{\tmthree}$,
and an {\em activation} operator $\str{\ctm}$.

The set of {\em free} and {\em bound variables} of a term or command is defined as expected.
The set of {\em free} and {\em bound names} of a term or command is defined as expected.
\end{marianaenv}
\end{definition}

\begin{marianaenv}
\subsection{Type system}
\begin{definition}[Types]
Suppose given a denumerably infinite set of {\em base types} $\set{\btyp, \btyptwo, \hdots}$.
The set of {\em types} is given by the following grammar:
\[
\begin{array}{lcl}
  \typ, \typtwo & ::=  & \btyp \mid \typ \iimp \typtwo \mid \typ \cimp \typtwo
\end{array}
\]
\mariana{  
Here $(\iimp)$ stands for the {\em intuitionistic} implication
whereas $(\cimp)$ stands for the {\em classical} implication.  
}
\end{definition}

\begin{definition}[Typing system]
Derivability of typing judgments $\judgE{\ctx}{O}{\typ}{\stoup}$,
where $O$ is a term or a command,
is defined inductively by the following rules.
\begin{figure}[htp]
\[
  \indrule{ax}{}{
   \judgE{\ctx, \var : \typ}{\var}{\typ}{\stoup}
  }
\]
\[
  \indrule{I-$\iimp$}{
    \judgE{\ctx,\var:\typ}{\tm}{\typtwo}{\stoup}
  }{
    \judgE{\ctx}{\absi{\var}{\tm}}{\typ \iimp \typtwo}{\stoup}
  }
  \HS
  \indrule{E-$\iimp$}{
    \judgE{\ctx}{\tm}{\typ \iimp \typtwo}{\stoup}
    \HS
    \judgE{\ctx}{\tmtwo}{\typ}{\stoup}
    \HS
    \judgE{\ctx, \var : \typtwo}{\tmthree}{\typthree}{\stoup}
  }{
    \judgE{\ctx}{\appi{\tm}{\tmtwo}{\var}{\tmthree}}{\typthree}{\stoup}
  }
\]
\[
%\mariana{
  \indrule{I-$\cimp$}{
    \judgE{\ctx,\var:\typ}{\ctm}{\bot}{\stoup \cup \set{\name : \typtwo}}
  }{
    \judgE{\ctx}{\absc{\var}{\name}{\ctm}}{\typ \cimp \typtwo}{\stoup}
  }
%}
  \HS
  \indrule{E-$\cimp$}{
    \judgE{\ctx}{\tm}{\typ \cimp \typtwo}{\stoup}
    \HS
    \judgE{\ctx}{\tmtwo}{\typ}{\stoup}
    \HS
    \judgE{\ctx, \var : \typtwo}{\ctm}{\bot}{\stoup}
  }{
    \judgE{\ctx}{\appc{\tm}{\tmtwo}{\var}{\ctm}}{\bot}{\stoup}
  }
\]
\[
  \indrule{der}{
    \judgE{\ctx}{\tm}{\typ}{\stoup}
  }{
    \judgE{\ctx}{\command{\name}{\tm}}{\bot}{\stoup \cup \set{\name : \typ}}
  }
  \HS
  \indrule{W$_i$}{
    \judgE{\ctx}{\ctm}{\bot}{\stoup}
  }{
    \judgE{\ctx}{\#c}{\typtwo}{\stoup}
  }
\]
\caption{Ecumenical type system for the $\lambdamulep$-calculus.}\label{fig:typesystem}
\end{figure}

\mariana{The constant type $\bot$ is used for typing commands.}
\end{definition}

\subsubsection{Annotations}
\begin{enumerate}
  \item Intuitionistic Weakening is not used for normalization, 
  but for provability of non minimal intuitionistic logic.
  
  {\bf Example:} $\stp{\neg B,A \cimp B, A}{C}{\cdot}$ is not provable:
  \[
  \infer[{\mbox{\red{\bf W}}}]{
    \stp{\Gamma}{C}{\cdot}
  }{
    \infer[{\cimp\mbox{-elim}}]{
      \stp{\Gamma}{\cdot}{\cdot}
    }{
      \infer[]{
        \stp{\Gamma}{A \cimp B}{\cdot}
      }{
        (A \cimp B \textit{ is in } \Gamma)
      }
      \HS
      \infer[]{
      \stp{\Gamma}{A}{\cdot}
      }{
        (A \textit{ is in } \Gamma)
      }
      \HS
      \infer[]{
      \stp{\Gamma,B}{\cdot}{\cdot}
      }{
        (\neg B \textit{ is in } \Gamma)
      }
    }
  }
  \]
  with $\Gamma = \neg B,A \cimp B,A$.
  The strategy of this proof is to remove $C$ from the stoup.
  This example indicates that the {\em reductio ad absurdum} can't be captured without having intuitionistic weakening.
  \item Instead of having commands without types, we could use $\bot$.
  \item La lógica clásica colapsa si tenemos:
    \[
      \indrule{}{
        \judgE{\ctx, \var:\typ}{\ctm}{\bot}{\name : \typtwo, \stoup}
      }{
        \judgE{\ctx, \var:\typ}{\mu \name.\ctm}{\typtwo}{\stoup}
      }
    \]
  \item We don't have $A \to \bot$, it's captured with $\cimp$ (and $A \iimp \bot \equiv A \cimp \bot$). 
  But it depends on the system used.
  \item In this example, $\str{\ctm}$ changes its type!
  (Note: this was fixed, I think)
  \[
    \indrule{}{
      \judgE{\ctx}{\str{\ctm}}{\typ \iimp \typtwo}{\stoup}
      \HS
      \judgE{\ctx}{\tmthree}{\typ}{\stoup}
      \HS
      \judgE{\ctx, \var:\typtwo}{\tmfour}{\typthree}{\stoup}
    }{
      \judgE{\ctx}{\appi{\str{\ctm}}{\tmthree}{\var}{\tmfour}}{\typthree}{\stoup}
    }
  \]
  \item\label{ite:obs} {\bf Observation (16/11/22):} Since types $((\typ \cimp \typtwo) \cimp \typ) \cimp \typ$
  and $((\typ \iimp \typtwo) \iimp \typ) \cimp \typ$ are equivalent,
  then the terms 
  $\absc{\var}{\name}{\appc{\var}{\absc{\vartwo}{\nametwo}{\command{\name}{\vartwo}}}{\vartwo}{\command{\name}{\vartwo}}}$
  and $\absc{\var}{\name}{\appc{\var}{\absi{\vartwo}{\str{\command{\name}{\vartwo}}}}{\vartwo}{\command{\name}{\vartwo}}}$
  should behave the same way.
\end{enumerate}

\subsection{Typing examples}
\begin{enumerate}
  \item {\bf Peirce's Law}: Using the observation~\ref{ite:obs} from the previous subsection
  (and the Example~\ref{ex:Peirce}), we have two terms with 
  equivalent types\footnote{four actually, as the types $((\typ \iimp \typtwo) \cimp \typ) \cimp \typ$ 
  and $((\typ \cimp \typtwo) \cimp \typ) \cimp \typ$ are equivalent to the previous two.}.
  \begin{enumerate}
    \item Let
    \[
    \deriv := \left(
    \indrule{I-$\cimp$}{
      \indrule{der}{
        \indrule{ax}{\emptyPremise}{\judgE{\ctx, \vartwo : \typ}{\vartwo}{\typ}{\nametwo : \typtwo}}
      }{
        \judgE{\ctx, \vartwo : \typ}{\command{\name}{\vartwo}}{\bot}{\name : \typ, \nametwo : \typtwo}
      }
    }{
      \judgE{\ctx}{\absc{\vartwo}{\nametwo}{\command{\name}{\vartwo}}}{\typ \cimp \typtwo}{\name : \typ}
    }
      \right)
    \]
    We build the following derivation:
    \[
    \indrule{I-$\cimp$}{
      \indrule{E-$\cimp$}{
        \indrule{ax}{
          \emptyPremise
        }{
          \judgE{
            \ctx
          }{
            {\var}
          }{
            (\typ \cimp \typtwo) \cimp \typ
          }{
            \name : \typ
          }
        }
        \HS
        \derivdots{\deriv}
        \HS
        \indrule{der}{
          \indrule{ax}{\emptyPremise}{\judgE{\ctx, \vartwo : \typ}{\vartwo}{\typ}{\estoup}}
        }{
          \judgE{\ctx, \vartwo : \typ}{\command{\name}{\vartwo}}{\bot}{\name : \typ}
        }
      }{\judgE{\ctx}{
          \appc{\var}{\absc{\vartwo}{\nametwo}{\command{\name}{\vartwo}}}{\vartwo}{\command{\name}{\vartwo}}
        }{\bot}{\name : \typ}}
    }{\judgE{\ectx}{
        \absc{\var}{\name}{\appc{\var}{\absc{\vartwo}{\nametwo}{\command{\name}{\vartwo}}}{\vartwo}{\command{\name}{\vartwo}}}
      }{((\typ \cimp \typtwo) \cimp \typ) \cimp \typ}{\estoup}
    }
    \]
    where $\ctx = \var : (\typ \cimp \typtwo) \cimp \typ$

    \item Let
    \[
    \deriv := \left(
    \indrule{I-$\iimp$}{
      \indrule{W$_i$}{
        \indrule{der}{
          \indrule{ax}{\emptyPremise}{\judgE{\ctx, \vartwo : \typ}{\vartwo}{\typ}{\nametwo : \typtwo}}
        }{
        \judgE{\ctx, \vartwo : \typ}{\command{\name}{\vartwo}}{\bot}{\name : \typ, \nametwo : \typtwo}
        }
      }{
        \judgE{\ctx, \vartwo : \typ}{\str{\command{\name}{\vartwo}}}{\typtwo}{\name : \typ, \nametwo : \typtwo}
      }
    }{
      \judgE{\ctx}{\absi{\vartwo}{\str{\command{\name}{\vartwo}}}}{\typ \iimp \typtwo}{\name : \typ}
    }
      \right)
    \]
    We build the following derivation:
    \[
    \indrule{I-$\cimp$}{
      \indrule{E-$\cimp$}{
        \indrule{ax}{\emptyPremise}{
          \judgE{\ctx}{{\var}}{(\typ \iimp \typtwo) \iimp \typ}{\name : \typ}
        }
        \HS
        \derivdots{\deriv}
        \HS
        \indrule{der}{
          \indrule{ax}{
            \emptyPremise
          }{
            \judgE{\ctx, \vartwo : \typ}{\vartwo}{\typ}{\estoup}
          }
        }{
          \judgE{\ctx, \vartwo : \typ}{\command{\name}{\vartwo}}{\bot}{\name : \typ}
        }
      }{\judgE{\ctx}{
        \appc{\var}{\absi{\vartwo}{\str{\command{\name}{\vartwo}}}}{\vartwo}{\command{\name}{\vartwo}}
      }{\bot}{\name : \typ}}
    }{\judgE{\ectx}{
        \absc{\var}{\name}{\appc{\var}{\absi{\vartwo}{\str{\command{\name}{\vartwo}}}}{\vartwo}{\command{\name}{\vartwo}}}
      }{
        ((\typ \iimp \typtwo) \iimp \typ) \cimp \typ
      }{
        \estoup
      }
    }
    \]
    where $\ctx = \var : (\typ \iimp \typtwo) \iimp \typ$
  \end{enumerate}
\end{enumerate}

\end{marianaenv}



\iffalse
The $\lambda\mu-\mathcal{LE}_p$ calculus has two kinds of variables: the $\lambda$-variables $x, y, z,\ldots$, and the $\mu$- variables $\alpha,\beta,\gamma,\ldots$. The $\lambda$-variables refer to the {\em stoup}, while the $\mu$-variables abstract the formulas in the classical context.

where $x,u,\lambda x.u,t,  t(u,y.r),\mu\alpha.e$ are {\em unnamed terms} and $e,[\alpha]t$ are {\em named terms}.  $t(u, y.r)$ is a generalized application, and it intuitively means that $t$ is applied to $u$ in the context of the substitution $\{\_/y\}r$.

\red{I have {\bf no clue} how to do it yet!!!}
Reductions:
\begin{itemize}
\item logical reduction:
\[\D\langle \lambda x.t\rangle(u,y.r)\to \{\D\langle\{u/x\}t\rangle/y\}r
\]
where $\D$ id a {\em distant context} as described in~\cite{DBLP:conf/fossacs/SantoKP22}.
\item  structural reduction: 
\[
(\mu\beta. e)(u,y.r) \to\mu\beta.e[([\beta]w)(u,y.r)/[\beta]w)]
\]
where $e[([\beta]w)(u,y.r)/[\beta]w)$ is obtained from $e$ by replacing inductively each subterm of the form $[\beta]w$ by $([\beta]w)(u,y.r)$.
\item renaming:
\[
[\alpha]\mu\beta.u\to u[\alpha/\beta]
\]
\end{itemize}



\section{TODO}
What we should prove (at least):
\begin{enumerate}
\item confluence;
\item subject reduction;
\item (strong?) normalization.
\end{enumerate}

Moreover: 
\begin{enumerate}
\item fragments;
\item mixed behaviors (classical and intuitionistic).
\item abstract state machine
\end{enumerate}
\fi