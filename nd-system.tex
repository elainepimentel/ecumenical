%!TEX root = Ecumenical-types.tex
% !TEX spellcheck = en-US

In this work, we will be mostly interested in the implicational fragment of ecumenical systems, so we will restrict our discussion to implications.\footnote{EP. It is still not clear to me if we really need to be minimal, so I will leave bottom and negation. Easier to drop them later, than to re-add.}

Sequents in $\LEi$ have the form $\stpg$, an extension of natural deduction sequents, where $\Pi$ is a multiset of at most one formula-- the {\em stoup} -- and  $\Gamma,\Delta$ are multisets of formulas -- the classical contexts.\footnote{As for the case of $\LC$~\cite{DBLP:journals/mscs/Girard91}, the {\em stoup} will carry the intuitionistic (positive) and neutral information, while the context accumulates the classical information related to it.}  

The (additive version of the) natural deduction system $\LEi$  is presented in Figure~\ref{fig:LEi}. 

\begin{figure}[ht]
{\sc Intuitionistic rules}
\[
\begin{array}{lc@{\qquad}l}
\infer[{\iimp\mbox{-int}}]{\stp{\Gamma}{A\iimp B}{\Delta}}{\stp{\Gamma,A}{B}{\Delta}}
& &
\infer[{\iimp\mbox{-elim}}]{\stp{\Gamma}{C}{\Delta}}{\stp{\Gamma}{A\iimp B}{\Delta} &
\stp{\Gamma}{A}{\Delta} & \stp{\Gamma,B}{C}{\Delta}}
\end{array}
\]
\\[5pt]
{\sc Classical rules}
\[
\begin{array}{lc@{\qquad}l}
\infer[{\cimp\mbox{-int}}]{\stp{\Gamma}{A\cimp B}{\Delta}}{\stp{\Gamma,A}{\cdot}{B,\Delta}}
& &
\infer[{\cimp\mbox{-elim}}]{\stp{\Gamma}{\cdot}{\Delta}}{\stp{\Gamma}{A\cimp B}{\Delta} &
\stp{\Gamma}{A}{\Delta} & \stp{\Gamma,B}{\cdot}{\Delta}}
\end{array}
\]
\\[5pt]
{\sc Neutral rules}
\[
\begin{array}{lc@{\qquad}l}
\infer[{\neg\mbox{-int}}]{\stp{\Gamma}{\neg A}{\Delta}}{\stp{\Gamma,A}{\cdot}{\Delta}}
& &
\infer[{\neg\mbox{-elim}}]{\stp{\Gamma}{\cdot}{\Delta}}{\stp{\Gamma}{A}{\Delta} &
\stp{\Gamma}{\neg A}{\Delta}}
\end{array}
\]
\\[5pt]
{\sc Hypothesis formation \& Dereliction}
\[
\begin{array}{lc@{\qquad}l}
\infer[\init]{\stp{\Gamma,A}{A}{\Delta}}{}
& &
\infer[\mbox{der}]{\stp{\Gamma}{B}{\Delta,A}}{\stp{\Gamma}{A}{\Delta,A}}
\end{array}
\]
\\[5pt]
{\sc Structural rule} \red{EP. We can drop this later}
\[
\begin{array}{lc@{\qquad}lc@{\qquad}l}
\infer[W_i]{\stp{\Gamma}{A}{\Delta}}{\stp{\Gamma}{\cdot}{\Delta}} 
\end{array}
\]
\caption{Ecumenical natural deduction system $\LEi$.}\label{fig:LEi}
\end{figure}

Derivations are  inductively defined in the usual way. 
\begin{definition}\label{def:der}
A $\LEi${\em derivation} is a finite rooted tree with nodes labeled by
sequents, axioms at the leaf nodes, and where each node is connected
with the (immediate) successor nodes (if any) according to its
inference rules. A sequent $S$ is \emph{derivable} in the sequent
system $\LEi$, notation $\deduc{\LEi}{S}$, iff there is a derivation
of $S$ in $\LEi$.  The system $LEi$ is usually omitted when it can
be inferred from the context.

A formula $A$ is a {\em theorem} of $\LEi$ if and only if $\vL \stp{}{A}{}$. 
\end{definition}

Finally, it is an easy matter to show that the implicational fragment of $\mathcal{NE}$ and $\LEi$ are equivalent.


\begin{example}[The Peirce's Law]
\label{ex:Peirce}
We observe that any sequent of the form  $(((A \to_{j} B) \to_{k} A) \to_{c} A)$ with $j,k\in\{i,c\}$ is provable in  $\LEi$. That is, provability is maintained if the outermost implication is classical. And more than that: all the proofs have the same following structure, where $\Gamma = (A\cimp B)\cimp A$.
\[
\infer[\cimp\mbox{-int}]{\stp{}{(((A\cimp B)\cimp A)\cimp A}{\cdot}}
{\infer[\cimp\mbox{-elim}]{\stp{\Gamma}{\cdot}{A}}
{\infer[\init]{\stp{\Gamma}{(A\cimp B)\cimp A}{A}}{}&
\infer[\cimp\mbox{-int}]{\stp{\Gamma}{A\cimp B}{A}}
{\infer[\der]{\stp{\Gamma,A}{B}{A}}
{\infer[\init]{\stp{\Gamma,A}{A}{A}}{}}}&
\infer[\der]{\stp{\Gamma,A}{\cdot}{A}}
{\infer[\init]{\stp{\Gamma,A}{A}{A}}{}}}}
\]
\end{example}

\red{EP. I am still thinking about minimal vs intuitionistic. More later!}
