%!TEX root = Ecumenical-types.tex
% !TEX spellcheck = en-US

The idea for normalization  of systems with stoup follows the usual one for natural deduction systems: show how to {\em compose} derivations, so to eliminate detours. The presence of {\em stoups}, however, adds an extra case analysis, since the composition may occur in the {\em stoup} or in the classical context. 
\subsection{Composition}
$\LEi$ has two modes of composition: a composition that occurs in the \textit{stoup} and a composition that occurs in the classical context. 

\subsection{Composition in the \textit{stoup}}
\begin{theorem}\label{thm:comp-stoup}
Let $\pi_{1}$ be a derivation of $\stp{\Gamma}{A}{\Delta}$ and $\pi_{2}$ be a derivation of $\stp{\Gamma,A}{B}{\Delta}$ in $\LEi$. Then, the result of replacing the assumption $A$ in $\pi_{2}$ by the derivation $\pi_{1}$ is a derivation $\pi$ of $\stp{\Gamma}{B}{\Delta}$. 
\end{theorem}


\subsection{Composition in the \textit{context}}
\begin{theorem}
Let $\pi_{1}$ be a derivation of $\stp{\Gamma}{C}{\Delta,A}$, and $\pi_{2}$ be a derivation of $\stp{\Gamma,A}{B}{\Delta}$. Then, the result of replacing the assumption $A$ in $\pi_{2}$ by the derivation $\pi_{1}$ is a derivation $\pi$ of $\stp{\Gamma}{C}{\Delta, B}$, where  $\Delta_{1}^{*}$ is obtained from $\Delta_{1}$ by means of the elimination of the occurrences of $A$.
\end{theorem}


In what follows, we shall use the following notation to indicate composition in the \textit{stoup}  and composition  in the \textit{classical context}.
\begin{itemize}
\item Composition in the \textit{stoup}:
\[
\deduce{\stp{\Gamma}{B}{\Delta}}
{\deduce{}
{\deduce{\pi_1[\frac{A}{A}]}
{\deduce{}
{\deduce{\stp{\Gamma}{A}{\Delta}}
{\deduce{}{\pi_2}}}}}}
\]


\item Composition in the \textit{classical context}:
\[
\deduce{\stp{\Gamma}{\cdot}{\Delta,B}}
{\deduce{}
{\deduce{\pi_1[\frac{A}{A}]}
{\deduce{}
{\deduce{\stp{\Gamma}{A}{\Delta,B}}
{\deduce{}{\pi_2}}}}}}
\]

\item A more concise notation for composition in general: $\pi_{2}/[\frac{A}{A}]/\pi_{1}$.

\end{itemize}

\subsection{Reductions}
Derivations in $\LEi$ may contain \emph{detours}. These detours are of two types: we may introduce a formula by an application of an introduction rule to immediately use it as major premiss of an elimination rule; or we may introduce a formula by an application of an introduction rule and use it as major premiss of an elimination rule after several application of $\iimpe$. The \emph{reductions} defined in this section are intended, as usual, to eliminate \emph{detours} that may occur in a derivation.


\begin{definition}
A derivation that begins with with the consequence of an application of an introduction rule to a formulas $A$ or $W_{i}$ and ends with an application of the respective elimination rule is called a {\em maximal segment}. A maximal segment of length 1 is called the {\em maximum formula}.
\end{definition}

\begin{definition}
Let $\Pi$ be a derivation in $\LEi$. The {\em degree of $\Pi$}, $d[\Pi ]$, is defined as $\max\{d[A]:$ A is the end-formula of maximal segment in $\Pi \}$, where $d[A]$ is the {\em weight} of the formula $A$, defined inductively by
$$
\begin{array}{lcl}
d[\bot]=d[p] &=& 0 \quad p\mbox{ atomic.}\\
d[A \to_{i,c} B] &=& d[A] + d[B] + 1 \\
d[\neg A] &=& d[A] +1.
\end{array}
$$
\end{definition}


\begin{definition}
A derivation $\Pi$ is called {\em normal} if and only if $d[\Pi ] = 0$.
\end{definition}
We will present next all the reduction steps in $\LEi$, used in the elimination of maximal segments.
\begin{enumerate}
\item $\to_{i}$-reduction:\\

The derivation
\[
\infer[\iimpe]{\stp{\Gamma}{C}{\Delta}}
{\infer[\iimpi]{\stp{\Gamma}{A\iimp B}{\Delta}}
{\deduce{\stp{\Gamma,A}{B}{\Delta}}{\pi_1}}&
\deduce{\stp{\Gamma}{A}{\Delta}}{\pi_2}&
\deduce{\stp{\Gamma,B}{C}{\Delta}}{\pi_3}}
\]
reduces to:
\[
\deduce{\stp{\Gamma}{C}{\Delta}}
{\deduce{}
{\deduce{\pi_3[\frac{B}{B}]}
{\deduce{}
{\deduce{\stp{\Gamma}{B}{\Delta}}
{\deduce{}
{\deduce{\pi_1[\frac{A}{A}]}
{\deduce{}
{\deduce{\stp{\Gamma}{A}{\Delta}}{\pi_2}}}}}}}}}
\]

Observe that the case for negation is similar and simpler.

\item $\to_{c}$-reduction:

The derivation
\[
\infer[\cimpe]{\stp{\Gamma}{\cdot}{\Delta}}
{\infer[\cimpi]{\stp{\Gamma}{A\cimp B}{\Delta}}
{\deduce{\stp{\Gamma,A}{\cdot}{\Delta,B}}{\pi_1}}&
\deduce{\stp{\Gamma}{A}{\Delta}}{\pi_2}&
\deduce{\stp{\Gamma,B}{\cdot}{\Delta}}{\pi_3}}
\]
reduces to:
\[
\deduce{\stp{\Gamma}{\cdot}{\Delta}}
{\deduce{}
{\deduce{\pi_3[\frac{B}{B}]}
{\deduce{}
{\deduce{\stp{\Gamma}{\cdot}{\Delta,B}}
{\deduce{}
{\deduce{\pi_1[\frac{A}{A}]}
{\deduce{}
{\deduce{\stp{\Gamma}{A}{\Delta,B}}{\pi_2}}}}}}}}}
\]

\item $W_i$-reductions:
  \begin{enumerate}
  \item $\iimp$-reduction:

    The derivation
    \[
      \infer[\iimpe]{\stp{\Gamma}{C}{\Delta}}
      {\infer[W_i]{\stp{\Gamma}{A\iimp B}{\Delta}}
        {\deduce{\stp{\Gamma}{\cdot}{\Delta}}{\pi_1}}&
        \deduce{\stp{\Gamma}{A}{\Delta}}{\pi_2}&
        \deduce{\stp{\Gamma, B}{C}{\Delta}}{\pi_3}}        
    \]
    reduces to:
    \[
      \infer[W_i]{\stp{\Gamma}{C}{\Delta}}
      {\deduce{\stp{\Gamma}{\cdot}{\Delta}}{\pi_1}}
    \]
    
  \item $\cimp$-reduction:

    The derivation
    \[
      \infer[\cimpe]{\stp{\Gamma}{\cdot}{\Delta}}
      {\infer[W_i]{\stp{\Gamma}{A\cimp B}{\Delta}}
        {\deduce{\stp{\Gamma}{\cdot}{\Delta}}{\pi_1}}&
        \deduce{\stp{\Gamma}{A}{\Delta}}{\pi_2}&
        \deduce{\stp{\Gamma,B}{\cdot}{\Delta}}{\pi_3}}
    \]
    reduces to:
    \[
      \deduce{\stp{\Gamma}{\cdot}{\Delta}}{\pi_1}
    \]

    The case for negation is similar.
  \end{enumerate}

\item permutative reduction

  The derivation
  \[
    \infer{\stp{\Gamma}{D}{\Delta}}
    {\infer[\iimpe]{\stp{\Gamma}{C}{\Delta}}
      {\deduce{\stp{\Gamma}{A\iimp B}{\Delta}}
        {\pi_1}&
        \deduce{\stp{\Gamma}{A}{\Delta}}{\pi_2}&
        \deduce{\stp{\Gamma,B}{C}{\Delta}}{\pi_3}}&
      \stp{\Gamma_1}{D_1}{\Delta}&
      \cdots&
      \stp{\Gamma_n}{D_n}{\Delta}}
  \]
  where $C$ is the major premiss of an elimination rule with minor premiss $D_1, \dots, D_n$, reduces to:
  \[\infer[\iimpe]{\stp{\Gamma}{D}{\Delta}}
    {\deduce{\stp{\Gamma}{A\iimp B}{\Delta}}
      {\pi_1}&
      \deduce{\stp{\Gamma}{A}{\Delta}}
      {\pi_2}&
      \infer{\stp{\Gamma, B}{D}{\Delta}}{
        \deduce{\stp{\Gamma, B}{C}{\Delta}}
        {\pi_3}&
        \stp{\Gamma_1, B}{D_1}{\Delta}&
        \dots&
        \stp{\Gamma_n, B}{D_n}{\Delta}}}
  \]
\end{enumerate}